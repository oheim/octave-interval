This chapter is contributed by Ch. Q. Lauter.

\section{Main considerations, critical accuracy bounds}\label{subsec:criticalboundslog10}
If one wants to guarantee that an implementation of the logarithm in
base $10$, $\log_{10}\left( x \right)$, in double precision is
correctly rounded, one has to ensure that the final intermediate
approximation before rounding to double precision has a relative error
of less than $2^{-122}$.

An implementation of $\log_{10}\left(x\right)$ can also be derived from an
implementation of the natural logarithm $\ln\left(x\right)$ using the formula:

\begin{equation}
  \log_{10}\left( x \right) = \frac{1}{\ln\left( 10 \right)} \cdot
\ln\left( x \right)\label{eq:log10}
\end{equation}
When doing so, one must ensure that the constant
$\mathit{log10inv} = \frac{1}{\ln\left(10\right)}$ is stored with
enough accuracy, that the approximation for $\ln\left( x \right)$ is
exact enough and that the multiplication sequence does not introduce 
too great an error. As we will see in the next section
\ref{subsec:outlinelog10}, this implies slight changes to the code for
the natural logarithm with regard to what has been presented in
 Chapter \ref{chap:log}.

 With regard to final rounding, the elementary function $\log_{10}$
 presents a particular issue that is somewhat singular amongst all
 considered elementary functions: There exist a large set of input
 double-precision numbers $x$ such that $\log_{10}(x)$ is a rational
 and is representable as a double-precision number.

 For such cases, a final directed rounding will be correct only if the
 approximation error is exactly $0$. 
 Indeed, the rounding $\diamond\left( f\left( x \right) \right)$ of
 the exactly representably value $f\left(x\right) \in \F$ is trivially
 $\diamond\left( f\left( x \right) \right) = f\left( x \right)$
 \cite{IEEE754}. In contrast, $\diamond\left( f\left( x \right) +
   \delta \right) \not = f\left( x \right) \in \F$ holds for all
 $\left \vert \delta \right \vert > 0$. 

 As it is impossible to achieve an approximation error exactly equal
 to zero, it is preferable to filter out such cases and handle them
 separately. Other functions described so far had only one such argument
 ($x=1$ for $\log$, x=0 for the trigs). $\log_2$ has a set of such
 cases ($x = 2^k$, $k \in \Z$) which is equally trivial to handle in
 binary floating point.
 

For $f = \log_{10}$, filtering much more difficult. In fact, $y =
\log\left( x \right)$ is algebraic for exactly all $x = 10^k$, $k \in
\Z$ \cite{Baker75}. Filtering means thus testing whether an input $x$
can be written $x = 10^k$ with an integer $k \in \Z$. This is
equivalent to testing if $\log_{10}\left( x \right)$ an integer,
i.e. $\log_{10}\left( x \right) \in \Z$. However, since $\log_{10}\left( x
\right)$ can only be approximated, filtering this way is
impossible.

One possibility is the following approach. In
floating point arithmetic, in order to be in a situation of difficult
rounding, not only $\log_{10}\left( x \right)$ must be algebraic but
also the corresponding $x = 10^k$, $k \in \Z$, must be representable
in floating point. To start with eliminating cases, we can argue that
this impossible for all $k < 0$. Indeed, since $2 \nmid 5$, there
exist no $m \in \N$ and $e \in \Z$ for any $k \in \Z^-$ such that
$10^k = 2^e \cdot m$ \cite{Muller97}. So we have reduced the range of
cases to filter to all $x = 10^k$, $k \in \N \cup \left \lbrace
0\right \rbrace$. Further in double precision, the mantissa's length
is $53$. So $10^k = 2^k \cdot 5^k = 2^e \cdot m$ is exactly
representable in double precision only for values $k \in \N \cup \left
\lbrace 0 \right \rbrace$ such that $5^k \leq 2^{53}$. This yields to
$k \leq 53 \cdot \frac{\ln\left( 2 \right)}{\ln\left( 5 \right)}
\approx 22.82$; hence $0 \leq k \leq 22$. In consequence, it would be
possible to filter out the $23$ arguments $x = 10^k$ with $k \in \left
\lbrace 0 \dots 22 \right \rbrace$. Nevertheless, this approach would
be relatively costly. It is not the way that has been chosen for the
implementation presented here.

Our approach uses the critical worst case accuracy of the
elementary function $\log_{10}\left( x \right)$. As already mentioned,
it is $2^{-122}$. Under the condition that we can provide an
approximation to the function that is exact to at least $2^{-123}$, we
can decide the directed rounding using a modified final rounding
sequence: We know that a $1$ after a long series of $0$s (respectively
a $0$ after a long series of $1$s) must be present at least at the
$122$th bit of the intermediate mantissa.  If it is not, we can
consider a potentially present $1$ after the $122$th bit to be an
approximation error artefact. In fact this means neglecting $\delta$s
relatively less than $2^{-122}$ when rounding $\diamond \left( f\left(
x \right) + \delta \right)$ instead of $\diamond \left( f\left( x
\right) \right)$. 

One shortcoming of this approach is that the
accurate phase is launched for arguments where the quick phase's
accuracy would suffice to return the correct result. As
such arguments are extremely rare ($p = \frac{23}{2^{63}} \approx 2.5
\cdot 10^{-18}$~!), this is not of an issue. The modification of the
final rounding sequence is relatively lightweight: merely one floating
point multiplication, two integer masks and one integer comparison
have to be added to handle the case.

One remarks this approach is only possible because the critical worst
case accuracy of the function is known by Lef{\`e}vre's works. Ziv's oignon peeling
strategy without the filtering of the
$23$ possible cases in input and without any accuracy limitation for
intermediate computations yields to nontermination of the
implementation of the function on such arguments $x = 10^k$.

An earlier \crlibm\ implementation of the $\log_{10}\left(x\right)$
function based on the SCS format did not handle the problem and
returned incorrectly rounded results for inputs $x = 10^k$ in the
directed rounding modes.


\section{General outline of the algorithm and accuracy estimates}\label{subsec:outlinelog10}
% 1/2 page
% 
% - Multiply by the right constant, this time using a triple double for the constant => Mul33 which is costly
% - Tell about the need to gain some bits in the log for the worst case => renormalize at some point in the code
% - Analyse the issue of integer powers of 10 => give explanation that there are only 17 cases 
% - Indicate the way the final rounding sequence for triple double can be modified => additional costs
% - Mention that we launch the accurate phase even for results where the quick phase result suffices (10^n), 
%   analyse the problem and mention that it is unique for log10 (in the usual list of elementary functions) 
%   but that it is quasi impossible to get around it (tell that log10 in SCS did not correctly treat the problem)

The quick phase of the implementation of the $\log_{10}\left( x
\right)$ follows exactly the scheme depicted by equation
(\ref{eq:log10}) above. Similarly to the logarithm in base
$2$, the natural logarithm's intermediate double-double result is
multiplied by a double-double precision approximation of
$\mathit{log10inv}$. The rounding test is slightly modified in order
to ensure safe rounding or launching the accurate phase.

Concerning the accurate phase, some modifications in the natural
logarithm's code are necessary because of the tighter accuracy bound
needed for the worst case. The natural logarithms accurate phase
polynomial approximation relative error has already been less than
$2^{-125}$ which is exact enough for $\log_{10}\left( x
\right)$. The fact that the complete triple-double implementation is
exact to only $119$ bits, is mainly due to the inexactness of the
operators used in reconstruction phase. In turn, this inexactness is
caused by the relatively high overlap in the triple-double numbers
handled. By adding two additional renormalisations the triple-double
operators become exact enough.

The constant $\mathit{log10inv}$ cannot be stored in double-double
precision with an accuracy of $124$ bits. A triple-double
approximation is therefore used. Its relative approximation error is
smaller than $2^{-159}$. The final multiplication of the triple-double
constant representing $\mathit{log10inv}$ and the triple-double
natural logarithm result is performed by a \MulTT. The relative error
of this operator on non-overlapping triple-doubles is not greater than
$2^{-140}$. This last operation therefore offers a large accuracy
overkill.

TODO The combination of the previous errors should be verified in Gappa.


\section{Timings}\label{subsec:timingslog10}

We compare \crlibm's portable triple-double implementation
for $\log_{10}\left( x \right)$ to other correctly rounded and
not-correctly rounded implementations.  ``\crlibm\ portable using
\scslib'' is the timing for the earlier implementation in {\tt
  crlibm}, which has been superseded by the one depicted here since
version 0.10$\beta$. This earlier implementation was completely based
on the SCS format and did not contain a quick phase implemented in
double precision arithmetic. The values are given in arbitrary units
and obtained on a IBM Power 5 processor with gcc 3.3.3 on a Linux
Kernel 2.6.5.
 
\begin{table}[h]
  \begin{center}
\begin{tabular}{|l|r|r|}
 \hline
  Library                       &     avg time  & max time \\
 \hline
 \hline
 \multicolumn{3}{|c|}{Power5 / Linux-2.6 / gcc-3.3}   \\ 
 \hline
 \texttt{MPFR}   &   9490    & 84478        \\ 
 \hline
 \crlibm\ portable using \texttt{scslib}   &   2624    & 2744        \\ 
 \hline
 \crlibm\ portable using triple-double      &        60    & 311        \\ 
 \hline
 default \texttt{libm} (not correctly rounded)   &        66    & 71      \\ 
 \hline
 \hline
 \multicolumn{3}{|c|}{PentiumM / Linux-2.6 / gcc-4.0}   \\ 
 \hline
 \crlibm\ portable using triple-double                  &        304    & 1529      \\ 
 \hline
 default \texttt{libm}  (not correctly rounded)          &        153    & 1904      \\ 
 \hline
 \hline
\end{tabular}
\end{center}
\caption{Log10 timings on Power5 and PentiumM architectures}
\label{Log10timings}
\end{table}

On average, our triple-double based implementation is even $10\%$
faster than its  incorrectly rounding counterpart on Power. On
Pentium, we observe the usual factor 2 with respect to an
implementation using double-extended arithmetic. Worst case timings
are acceptable in both cases.



%%% Local Variables: 
%%% mode: latex
%%% TeX-master: "crlibm"
%%% End: 
